\documentclass[a4paper, 12pt]{article}
\usepackage[utf8]{inputenc}
\usepackage{polski}
\usepackage{hyperref}
\usepackage{graphicx}
\usepackage{indentfirst}
\usepackage{float}

\title{Metody odkrywania wiedzy -- wykrywanie ataków sieciowych}
\author{Adam Stelmaszczyk\\ Anna Stępień}

\begin{document}

\maketitle

\tableofcontents

\newpage

\section{Interpretacja tematu projektu}
Projekt ma na celu analizę działania różnych algorytmów klasyfikacji zastosowanych
w procesie wykrywania ataków sieciowych. Połączenia sieciowe można podzielić na dwa typy:
\begin{itemize}
	\item prawidłowe,
	\item nieprawidłowe, naruszające bezpieczeństwo.
\end{itemize}
W ramach projektu zbadane zostaną algorytmy, które opisano w sekcji \ref{algorithms} 
tej pracy.

\section{Opis wykorzystanych algorytmów}\label{algorithms}

\subsection{Naiwny klasyfikator Bayesa}

Klasyfikator Bayesa wybiera klasę, która jest najbardziej prawdopodna dla danego przykładu na podstawie prawdopodobieństwa
warunkowego $P(c \mid x)$, gdzie $c$ oznacza etykietę klasy, a $x$ jest przykładem. Jest to tzw. prawdopodobieństwo
\textit{a posteriori}. Wyznaczane jest ze wzoru:

$$ P(c \mid x) = \frac{P(x \mid c)P(c)}{P(x)} $$

Przy poszukiwaniu maksimum $P(c \mid x)$ mianownik $P(x)$ można pominąć bez zmiany wyniku maksymalizacji.
Prawdopodobieństwo \textit{a posteriori} $P(c)$ można oszacować na podstawie częstości występowania klas w zbiorze treningowym.
Natomiast prawdopodobieństwo $P(x \mid c)$ jest szacowane ze wzoru:

$$ P(x \mid c) = \prod_{i=1}^n P(a_i(x) \mid c)$$

gdzie $a_i(x)$ oznacza $i$-ty atrybut przykładu $x$.
Przyjęte zostało założenie, że wartości poszczególnych atrybutów są niezależne od siebie.
Na tym polega ,,naiwność'' klasyfikatora Bayesa.

Naiwny klasyfikator Bayesa zostanie przetestowany przy pomocy funkcji \texttt{naiveBayes} z~pakietu \texttt{e1071} języka~R
\footnote{\url{http://www-users.cs.york.ac.uk/~jc/teaching/arin/R_practical/}}.

\subsection{k-NN}
k-NN jest algorytmem, który dokonuje klasyfikacji na podstawie podobieństwa do k najbardziej podobnych przykładów ze zbioru uczącego. W~celu oszacowania podobieństwa poszczególnych próbek, stosowane są miary odległości takie jak np. metryka euklidesowa lub Manhattan. Klasa, która przypisywana jest do badanego obiektu ustalana jest na podstawie klasy reprezentowanej przez największą spośród wybranych wcześniej k przykładów.

Algorytm k-NN zostanie przetestowany z~wykorzystaniem funkcji \texttt{knn} oferowanej przez pakiet \texttt{FNN} języka~R \footnote{\url{http://cran.r-project.org/web/packages/FNN/}}.

\subsection{SVM}
SVM jest klasyfikatorem binarnym, który ma na celu wyznaczenie hiperpłaszczyzny rozdzielającej z~maksymalnym marginesem przykłady należące do dwóch klas. W~ramach projektu realizowana będzie klasyfikacja mająca na celu przypisanie połączeń sieciowych do jednej z~pięciu klas. Problem klasyfikacji wieloklasowej może być zdekomponowany do szeregu problemów klasyfikacji binarnej. W~projekcie wykorzystane zostanie podejście \textit{one-against-one}, które polega na zbudowaniu klasyfikatora SVM dla każdej pary klas. 

Do przetestowania klasyfikatora SVM zostanie wykorzystana funkcja \texttt{svm} z~pakietu \texttt{e1071} języka~R \footnote{\url{http://cran.r-project.org/web/packages/e1071/}}.

\section{Plan badań}

\subsection{Pytania, na które będzie poszukiwana odpowiedź}

\begin{enumerate}
 \item Jakie parametry dla algorytmów dadzą najlepsze wyniki?
 \item Który algorytm przy najlepszych parametrach da najlepsze wyniki?
 \item Który algorytm przy najlepszych parametrach będzie działał najszybciej?
 \item Które cechy miały największy wpływ na klasyfikację?

Wykorzystana zostanie funkcja \texttt{importance} z pakietu lasów losowych \texttt{randomForest}.
\footnote{\url{http://cran.r-project.org/web/packages/randomForest/randomForest.pdf}}
Funkcja ta zwraca wartości dwóch miar ważności dla wszystkich atrybutów.
Pierwszą miarą jest średni spadek precyzji, drugą jest średni spadek nieczystości węzłów.
\end{enumerate}

\subsection{Analiza danych wejściowych}

Zbiorem danych, który zostanie wykorzystany podczas realizacji projektu jest zbiór danych dostępny 
pod adresem: \\
\url{www.sigkdd.org/kdd-cup-1999-computer-network-intrusion-detection}.

Wykorzystywany zbiór danych składa się ze zbiorów: treningowego i testowego.
Każdy przykład ma 42 atrybuty.
Zbiór testowy charakteryzuje się różnym rozkładem prawdopodobieństwa niż zbiór treningowy oraz
zawiera więcej typów ataków sieciowych, które nie znajdują się w zbiorze treningowym.
Na podstawie cech, połączenia zostaną zaklasyfikowane do jednej z pięciu klas:

\begin{enumerate}
 \item \texttt{normal} -- połączenia prawidłowe.
 \item \texttt{probe} -- odpytywanie, np. skanowanie portów.
 \item \texttt{DOS} -- ataki typu Denial of Service.
 \item \texttt{U2R} -- przejęcie praw administratora lokalnej maszyny, np. ataki przepełnienia bufora.
 \item \texttt{R2L} -- nieautoryzowany dostęp ze zdalnej maszyny, np. zgadywanie hasła.
\end{enumerate}

Na początku dane zostaną przejrzane w poszukiwaniu pojedynczych wartości odstających
powstałych np. na skutek błędu ludzkiego. Następnie zostanie wybranych kilka najlepiej rokujących
cech. Przetestowane zostanie też działanie algorytmów biorące wszystkie cechy pod uwagę.
Dane zostaną znormalizowane -- dokonamy konwersji atrybutów nieliczbowych na liczbowe,
a~następnie przeskalujemy wszystkie wartości, tak aby miały taki sam rząd wielkości.

\subsection{Parametry algorytmów, których wpływ na wyniki będzie badany}

\begin{itemize}
 \item Wartość $k$ dla algorytmu k-NN.
 \item Rodzaj jądra dla SVM.
\end{itemize}

%TODO astepien dopisanie parametrów, które oferują funkcje z R, a które będziemy dostrajać

\subsection{Sposób oceny jakości modelu}
Jakość modelu zostanie poddana ocenie zgodnie z procedurą konkursową,
poprzez obliczenie kosztu pomyłek przy pomocy podanej macierzy kar:

\begin{table}[H]
\centering
\begin{tabular}{ l l l l l l }
       & \texttt{normal}&\texttt{probe}	&\texttt{DOS}	&\texttt{U2R}	&\texttt{R2L} \\
\texttt{normal}	&0	&1	&2	&2	&2 \\
\texttt{probe}	&1	&0	&2	&2	&2 \\
\texttt{DOS}	&2	&1	&0	&2	&2 \\
\texttt{U2R}	&3	&2	&2	&0	&2 \\
\texttt{R2L}	&4	&2	&2	&2	&0 \\
\end{tabular}
\caption{Macierz kar}
\label{table:cov_matrix}
\end{table}

Wykorzystywane w~projekcie dane charakteryzują się nierównomiernym rozkładem klas. W~zależności od otrzymywanych błędów klasyfikacji, konieczne może okazać się dokonanie pewnych modyfikacji w~procedurze przygotowania danych treningowych oraz w~samej postaci macierzy kar. 
\begin{itemize}
	\item strojenie parametrów przedstawionej macierzy kar, polegające na zwiększaniu wag dla klas znajdujących się w~zdecydowanej mniejszości w~stosunku do pozostałych klas, a~zmniejszanie ich dla klas, których rozkład jest równomierny.
	\item resampling przykładów treningowych w~celu uzyskania równomiernego rozkładu klas. Do tego celu wykorzystany algorytm \texttt{SMOTE} udostępniany przez pakiet \texttt{DMwR} \footnote{\url{http://cran.r-project.org/web/packages/DMwR/}} języka~R.
\end{itemize}

% Ponadto zostanie wykorzystana graficzna reprezentacja zależności wyników typu \textit{true-positive} i \textit{false-positive} w postaci krzywej ROC.

\end{document}