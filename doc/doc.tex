\documentclass[a4paper, 12pt]{article}
\usepackage[utf8]{inputenc}
\usepackage{polski}
\usepackage{hyperref}
\usepackage{graphicx}
\usepackage{indentfirst}
\usepackage{float}
\usepackage{amsfonts}

\title{Metody odkrywania wiedzy -- wykrywanie ataków sieciowych}
\author{Adam Stelmaszczyk\\ Anna Stępień}

\begin{document}

\maketitle

\tableofcontents

\newpage

\section{Interpretacja tematu projektu}
Projekt ma na celu analizę działania różnych algorytmów klasyfikacji zastosowanych
w procesie wykrywania ataków sieciowych. Połączenia sieciowe można podzielić na dwa typy:
\begin{itemize}
	\item prawidłowe (klasa \texttt{normal})
	\item nieprawidłowe, naruszające bezpieczeństwo
(klasy \texttt{probe}, \texttt{DOS}, \texttt{U2R}, \texttt{R2L})
\end{itemize} 

\section{Opis wykorzystanych algorytmów}\label{algorithms}

\subsection{k-NN}
k-NN jest algorytmem, który dokonuje klasyfikacji na podstawie podobieństwa do k 
najbardziej podobnych przykładów ze zbioru uczącego. W~celu oszacowania podobieństwa 
poszczególnych próbek, stosowane są miary odległości takie jak np. metryka euklidesowa lub Manhattan. 
Klasa, która przypisywana jest do badanego obiektu ustalana jest na podstawie klasy 
reprezentowanej przez największą spośród wybranych wcześniej k przykładów.

Algorytm k-NN zostanie przetestowany z~wykorzystaniem funkcji 
\texttt{knn} oferowanej przez pakiet \texttt{FNN} języka~R 
\footnote{\url{http://cran.r-project.org/web/packages/FNN/}}.

\subsection{Naiwny klasyfikator Bayesa}

Klasyfikator Bayesa wybiera klasę, która jest najbardziej prawdopodna dla danego przykładu na 
podstawie prawdopodobieństwa
warunkowego $P(c \mid x)$, gdzie $c$ oznacza etykietę klasy, a $x$ jest przykładem. 
Jest to tzw. prawdopodobieństwo
\textit{a posteriori}. Wyznaczane jest ze wzoru:

$$ P(c \mid x) = \frac{P(x \mid c)P(c)}{P(x)} $$

Przy poszukiwaniu maksimum $P(c \mid x)$ mianownik $P(x)$ można pominąć bez zmiany wyniku 
maksymalizacji.
Prawdopodobieństwo \textit{a posteriori} $P(c)$ można oszacować na podstawie częstości 
występowania klas w zbiorze treningowym.
Natomiast prawdopodobieństwo $P(x \mid c)$ jest szacowane ze wzoru:

$$ P(x \mid c) = \prod_{i=1}^n P(a_i(x) \mid c)$$

gdzie $a_i(x)$ oznacza $i$-ty atrybut przykładu $x$.
Przyjęte zostało założenie, że wartości poszczególnych atrybutów są niezależne od siebie.
Na tym polega ,,naiwność'' klasyfikatora Bayesa.

Naiwny klasyfikator Bayesa zostanie przetestowany przy pomocy funkcji \texttt{naiveBayes} z~pakietu \texttt{e1071} języka~R
\footnote{\url{http://www-users.cs.york.ac.uk/~jc/teaching/arin/R_practical/}}.

\subsection{SVM}
SVM jest klasyfikatorem binarnym, który ma na celu wyznaczenie hiperpłaszczyzny rozdzielającej 
z~maksymalnym marginesem przykłady należące do dwóch klas. W~ramach projektu realizowana będzie 
klasyfikacja mająca na celu przypisanie połączeń sieciowych do jednej z~pięciu klas. 
Problem klasyfikacji wieloklasowej może być zdekomponowany do szeregu problemów klasyfikacji binarnej. 
W~projekcie wykorzystane zostanie podejście \textit{one-against-one}, 
które polega na zbudowaniu klasyfikatora SVM dla każdej pary klas. 

Do przetestowania klasyfikatora SVM zostanie wykorzystana funkcja \texttt{svm} z~pakietu 
\texttt{e1071} języka~R \footnote{\url{http://cran.r-project.org/web/packages/e1071/}}.

\section{Plan badań}

\subsection{Pytania, na które będzie poszukiwana odpowiedź}

\begin{enumerate}
 \item Jakie parametry dla algorytmów dadzą najlepsze wyniki?
 \item Który algorytm przy najlepszych parametrach da najlepsze wyniki?
 \item Który algorytm przy najlepszych parametrach będzie działał najszybciej?
 \item Które cechy miały największy wpływ na klasyfikację?

Wykorzystana zostanie funkcja \texttt{importance} z pakietu lasów losowych \texttt{randomForest}.
\footnote{\url{http://cran.r-project.org/web/packages/randomForest/randomForest.pdf}}
Funkcja ta zwraca wartości dwóch miar ważności dla wszystkich atrybutów.
Pierwszą miarą jest średni spadek precyzji, drugą jest średni spadek nieczystości węzłów.
\end{enumerate}

\subsection{Analiza danych wejściowych}

Zbiorem danych, który zostanie wykorzystany podczas realizacji projektu jest zbiór danych dostępny 
pod adresem: \\
\url{www.sigkdd.org/kdd-cup-1999-computer-network-intrusion-detection}.

Wykorzystywany zbiór danych składa się ze zbiorów: treningowego i testowego.
Każdy przykład ma 42 atrybuty: \\

\begin{tabular}{ | c | p{3cm} | p{3cm} | p{6cm} | } \hline
$i$ & Nazwa & Opis & Zbiór wartości \\ \hline
0      & duration & length (number of seconds) of the connection & $\mathbb{Z}$ \\ \hline
1      & protocol\_type & type of the protocol & icmp, tcp, udp \\ \hline
2      & service & network service on the destination & smtp, bgp, imap4, courier, name, exec, ftp, echo, http\_2784,
                       http\_443, discard, kshell, login, http, Z39\_50, vmnet, supdup,
                       gopher, printer, aol, tftp\_u, csnet\_ns, http\_8001, eco\_i, time,
                       ssh, efs, hostnames, X11, klogin, sql\_net, ldap, private,
                       auth, uucp, pm\_dump, link, ctf, IRC, ecr\_i, netbios\_ns, urp\_i,
                       pop\_2, pop\_3, rje, systat, ftp\_data,finger, tim\_i, remote\_job,
                       other, domain\_u, urh\_i, iso\_tsap, netstat, daytime, whois, shell,
                       mtp, sunrpc, uucp\_path, red\_i, harvest, nnsp, telnet, domain,
                       ntp\_u, netbios\_dgm, nntp, netbios\_ssn \\ \hline
3      & flag & normal or error status of the connection  & REJ, SF, SH, RSTO, OTH, RSTR, RSTOS0, S0, S1, S2, S3 \\ \hline
4      & src\_bytes  & number of data bytes from source to destination  & $\mathbb{Z}$ \\ \hline
5      & dst\_bytes  & number of data bytes from destination to source  & $\mathbb{Z}$ \\ \hline
6      & land & 1 if connection is from/to the same host/port; 0 otherwise  & 0, 1 \\ \hline
7      & wrong\_fragment  & number of ``wrong'' fragments  & $\mathbb{Z}$ \\ \hline
8      & urgent  & number of urgent packets   & $\mathbb{Z}$ \\ \hline
\end{tabular}

\begin{tabular}{ | c | l | p{6,5cm} | p{1,5cm} | } \hline
$i$ & Nazwa & Opis & Zbiór wartości \\ \hline
8      & hot & number of ``hot'' indicators  & $\mathbb{Z}$ \\ \hline
9      & num\_failed\_logins  & number of failed login attempts   & $\mathbb{Z}$ \\ \hline
10      & logged\_in  & 1 if successfully logged in; 0 otherwise   & $\mathbb{Z}$ \\ \hline
11      & num\_compromised & number of ``compromised'' conditions   &  0, 1 \\ \hline
12      & root\_shell  & 1 if root shell is obtained; 0 otherwise  &  0, 1 \\ \hline
13      & su\_attempted  & 1 if ``su root'' command attempted; 0 otherwise &  0, 1 \\ \hline
14      & num\_root  & number of ``root'' accesses   & $\mathbb{Z}$ \\ \hline
15      & num\_file\_creations  & number of file creation operations   & $\mathbb{Z}$ \\ \hline
16      & num\_shells  & number of shell prompts  & $\mathbb{Z}$ \\ \hline
17      & num\_access\_files  & number of operations on access control files  & $\mathbb{Z}$ \\ \hline
18      & num\_outbound\_cmds & number of outbound commands in an ftp session   & $\mathbb{Z}$ \\ \hline
19      & is\_hot\_login  & 1 if the login belongs to the ``hot'' list; 0 otherwise   &  0, 1 \\ \hline
20      & is\_guest\_login  & 1 if the login is a ``guest''login; 0 otherwise  &  0, 1 \\ \hline
21     & count & number of connections to the same host as the current connection in the past two seconds  & $\mathbb{Z}$ \\ \hline
\end{tabular}

\begin{tabular}{ | c | l | p{4,5cm} | p{4cm} | } \hline
$i$ & Nazwa & Opis & Zbiór wartości \\ \hline
22     & srv\_count   & number of connections to the same service as the current connection in the past two seconds  &  $\mathbb{Z}$ \\ \hline
23      & serror\_rate   & \% of connections that have ``SYN'' errors    & $\mathbb{Z}$ \\ \hline
24      & srv\_serror\_rate   & \% of connections that have ``SYN'' errors    & $\mathbb{Z}$ \\ \hline
25      & rerror\_rate   & \% of connections that have ``REJ'' errors   & $\mathbb{Z}$ \\ \hline
26      & srv\_rerror\_rate   & \% of connections that have ``REJ'' errors   & $\mathbb{Z}$ \\ \hline
27      & same\_srv\_rate  & \% of connections to the same service   &  $\mathbb{Z}$ \\ \hline
28     & diff\_srv\_rate  & \% of connections to different services  &  $\mathbb{Z}$ \\ \hline  
29      & srv\_diff\_host\_rate   & \% of connections to different hosts   & $\mathbb{Z}$ \\ \hline
30     & dst\_host\_count & ?   &  $\mathbb{Z}$ \\ \hline
31      & dst\_host\_srv\_count  & ?    & $\mathbb{Z}$ \\ \hline
32      & dst\_host\_same\_srv\_rate  & ?    & $\mathbb{Z}$ \\ \hline
33      & dst\_host\_diff\_srv\_rate  & ?  & $\mathbb{Z}$ \\ \hline
34      & dst\_host\_same\_src\_port\_rate & ?   & $\mathbb{Z}$ \\ \hline
35      & dst\_host\_srv\_diff\_host\_rate & ?  &  $\mathbb{Z}$ \\ \hline
36     &  dst\_host\_serror\_rate       & ?      &  $\mathbb{Z}$ \\ \hline  
37      & dst\_host\_srv\_serror\_rate & ? & $\mathbb{Z}$ \\ \hline
38     & dst\_host\_srv\_serror\_rate   & ?   & $\mathbb{Z}$ \\ \hline
39      & dst\_host\_rerror\_rate   & ?  & $\mathbb{Z}$ \\ \hline
40      & dst\_host\_srv\_rerror\_rate & ?   & $\mathbb{Z}$ \\ \hline
41      & attack\_type & attack type  & back, buffer\_overflow, ftp\_write, guess\_passwd, imap, ipsweep, land, loadmodule, multihop, neptune, nmap, normal, perl, phf, pod, portsweep, rootkit, satan, smurf, spy, teardrop, warezclient, warezmaster \\ \hline
\end{tabular} \\\\

Zbiór testowy charakteryzuje się różnym rozkładem prawdopodobieństwa niż zbiór treningowy oraz
zawiera więcej typów ataków sieciowych, które nie znajdują się w zbiorze treningowym.
Na podstawie cech, połączenia zostaną zaklasyfikowane do jednej z pięciu klas:

\begin{enumerate}
 \item \texttt{normal} -- połączenia prawidłowe
 \item \texttt{probe} -- odpytywanie, np. skanowanie portów
 \item \texttt{DOS} -- ataki typu Denial of Service
 \item \texttt{U2R} -- przejęcie praw administratora lokalnej maszyny, np. ataki przepełnienia bufora
 \item \texttt{R2L} -- nieautoryzowany dostęp ze zdalnej maszyny, np. zgadywanie hasła
\end{enumerate}

Na początku dane zostaną przejrzane w poszukiwaniu pojedynczych wartości odstających
powstałych np. na skutek błędu ludzkiego. Następnie zostanie wybranych kilka najlepiej rokujących
cech. Przetestowane zostanie też działanie algorytmów biorące wszystkie cechy pod uwagę.
Dane zostaną znormalizowane -- dokonamy konwersji atrybutów nieliczbowych na liczbowe,
a~następnie przeskalujemy wszystkie wartości, tak aby miały taki sam rząd wielkości.

\subsection{Parametry algorytmów, których wpływ na wyniki będzie badany}
Poniżej przedstawiono parametry algorytmów, które będą analizowane.

\subsubsection*{k-NN}
\begin{itemize}
	\item wartość parametru $k$
	\item algorytm wyszukiwania najbliższego sąsiada
\end{itemize}

\subsubsection*{Naiwny klasyfikator Bayesa}
\begin{itemize}
	\item wygładzanie Laplace'a
\end{itemize}

\subsubsection*{SVM}
\begin{itemize}
 \item rodzaj jądra oraz parametry charakterystyczne dla poszczególnych typów
 \item wagi poszczególnych klas
\end{itemize}

\subsection{Sposób oceny jakości modelu}
Jakość modelu zostanie poddana ocenie zgodnie z procedurą konkursową,
poprzez obliczenie kosztu pomyłek przy pomocy podanej macierzy kar:

\begin{table}[H]
\centering
\begin{tabular}{ l l l l l l }
       & \texttt{normal}&\texttt{probe}	&\texttt{DOS}	&\texttt{U2R}	&\texttt{R2L} \\
\texttt{normal}	&0	&1	&2	&2	&2 \\
\texttt{probe}	&1	&0	&2	&2	&2 \\
\texttt{DOS}	&2	&1	&0	&2	&2 \\
\texttt{U2R}	&3	&2	&2	&0	&2 \\
\texttt{R2L}	&4	&2	&2	&2	&0 \\
\end{tabular}
\caption{Macierz kar}
\label{table:cov_matrix}
\end{table}

Wykorzystywane w~projekcie dane charakteryzują się nierównomiernym rozkładem klas. W~zależności od otrzymywanych błędów klasyfikacji, konieczne może okazać się dokonanie pewnych modyfikacji w~procedurze przygotowania danych treningowych oraz w~samej postaci macierzy kar. 
\begin{itemize}
	\item strojenie parametrów przedstawionej macierzy kar, polegające na zwiększaniu wag dla klas znajdujących się w~zdecydowanej mniejszości w~stosunku do pozostałych klas, a~zmniejszanie ich dla klas, których rozkład jest równomierny.
	\item resampling przykładów treningowych w~celu uzyskania równomiernego rozkładu klas. Do tego celu wykorzystany algorytm \texttt{SMOTE} udostępniany przez pakiet \texttt{DMwR} \footnote{\url{http://cran.r-project.org/web/packages/DMwR/}} języka~R.
\end{itemize}

% Ponadto zostanie wykorzystana graficzna reprezentacja zależności wyników typu \textit{true-positive} i \textit{false-positive} w postaci krzywej ROC.

\end{document}