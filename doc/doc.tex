\documentclass[a4paper, 12pt]{article}
\usepackage [MeX] {polski}
\usepackage [utf8] {inputenc}
\usepackage {hyperref}
\usepackage{graphicx}
\usepackage[nottoc,numbib]{tocbibind}
\usepackage{color}
\usepackage{indentfirst}

\title {Metody odkrywania wiedzy - wykrywanie ataków sieciowych}
\author {Adam Stelmaszczyk\\ Anna Stępień}

\begin{document}

\maketitle

\tableofcontents

\newpage

\section{Opis projektu}
Projekt ma na celu analizę działania różnych algorytmów klasyfikacji zastosowanych
w procesie wykrywania ataków sieciowych. Połączenia sieciowe można podzielić na dwa typy:
\begin{itemize}
	\item prawidłowe,
	\item nieprawidłowe, naruszające bezpieczeństwo.
\end{itemize}
W ramach projektu zbadane zostaną algorytmy, które opisano w sekcji \ref{algorithms} niniejszego dokumentu.

\section{Dane}
Zbiorem danych, który zostanie wykorzystany podczas realizacji projektu jest zbiór danych dostępny pod adresem http://www.sigkdd.org/kdd-cup-1999-computer-network-intrusion-detection

Wykorzystywany zbiór danych składa się ze zbiorów: treningowego i testowego. Zbiór testowy charakteryzuje się różnym rozkładem prawdopodobieństwa niż zbiór treningowy oraz zawiera więcej typów ataków sieciowych, które nie znajdują się w zbiorze treningowym.

Zdefiniowane w zbiorach danych typy ataków sieciowych mogą przynależeć do jednej z czterech kategorii: DOS, R2L, U2L, probing.

Na podstawie informacji na temat przynależności typów ataków sieciowych do poszczególnych klas, dane zostaną podzielone na pięć grup:
\begin{itemize}
	\item normal - połączenia prawidłowe,
	\item DOS - połączenia będące atakami typu DOS,
	\item R2L - połączenie będące atakami typu R2L,
	\item U2L - połączenia będące atakami typu U2L,
	\item probing - połączenia będące atakami typu probing.
\end{itemize}

%TODO napisać czy będziemy coś robić z atrybutami

\section{Analizowane algorytmy klasyfikacji}\label{algorithms}
% Mamy w sumie 5 klas, więc może być trochę więcej roboty niż z klasyfikatorem binarnym. Może wystarczą 2 algorytmy: SVM(jest sporo parametrów do wyboru) i jeszcze jakiś jeden?

\section{Sposób oceny jakości modelu}
Jakość modelu zostanie poddana ocenie poprzez analizę jakości klasyfikacji i macierzy pomyłek. Ponadto zostanie wykorzystana graficzna reprezentacja zależności wyników typu \textit{true-positive} i \textit{false-positive} w postaci krzywej ROC.

\end{document}