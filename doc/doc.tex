\documentclass[a4paper, 12pt]{article}
\usepackage[utf8]{inputenc}
\usepackage{polski}
\usepackage{hyperref}
\usepackage{graphicx}
\usepackage{indentfirst}
\usepackage{float}
\usepackage{amsfonts}

\title{Metody odkrywania wiedzy -- wykrywanie ataków sieciowych}
\author{Adam Stelmaszczyk\\ Anna Stępień}

\begin{document}

\maketitle

\tableofcontents

\newpage

\section{Interpretacja tematu projektu}
Projekt ma na celu analizę działania różnych algorytmów klasyfikacji zastosowanych
w procesie wykrywania ataków sieciowych. Połączenia sieciowe można zaklasyfikować do jednej
z pięciu klas:
\begin{itemize}
  \item prawidłowe (klasa \texttt{normal}),
  \item odpytywanie, np. skanowanie portów (klasa \texttt{probe}),
  \item ataki typu Denial of Service (klasa \texttt{dos}),
  \item ataki typu User to Root, polegające na przejęciu praw administratora, 
np. ataki przepełnienia bufora (klasa \texttt{u2r}),
  \item ataki typu Remote to Local, polegające na nieautoryzowanym dostępie ze zdalnego komputera 
(klasa \texttt{r2l}).
\end{itemize} 

\section{Opis wykorzystanych algorytmów}\label{algorithms}

\subsection{k-NN}
k-NN jest algorytmem, który dokonuje klasyfikacji na podstawie podobieństwa do k 
najbardziej podobnych przykładów ze zbioru uczącego. W~celu oszacowania podobieństwa 
poszczególnych próbek, stosowane są miary odległości takie jak np. metryka euklidesowa lub Manhattan. 
Klasa, która przypisywana jest do badanego obiektu ustalana jest na podstawie klasy 
reprezentowanej przez największą spośród wybranych wcześniej k przykładów.

Algorytm k-NN zostanie przetestowany z~wykorzystaniem funkcji 
\texttt{knn} oferowanej przez pakiet \texttt{FNN} języka~R 
\footnote{\url{http://cran.r-project.org/web/packages/FNN/}}.

\subsection{Naiwny klasyfikator Bayesa}

Klasyfikator Bayesa wybiera klasę, która jest najbardziej prawdopodna dla danego przykładu na 
podstawie prawdopodobieństwa
warunkowego $P(c \mid x)$, gdzie $c$ oznacza etykietę klasy, a $x$ jest przykładem. 
Jest to tzw. prawdopodobieństwo
\textit{a posteriori}. Wyznaczane jest ze wzoru:

$$ P(c \mid x) = \frac{P(x \mid c)P(c)}{P(x)} $$

Przy poszukiwaniu maksimum $P(c \mid x)$ mianownik $P(x)$ można pominąć bez zmiany wyniku 
maksymalizacji.
Prawdopodobieństwo \textit{a posteriori} $P(c)$ można oszacować na podstawie częstości 
występowania klas w zbiorze treningowym.
Natomiast prawdopodobieństwo $P(x \mid c)$ jest szacowane ze wzoru:

$$ P(x \mid c) = \prod_{i=1}^n P(a_i(x) \mid c)$$

gdzie $a_i(x)$ oznacza $i$-ty atrybut przykładu $x$.
Przyjęte zostało założenie, że wartości poszczególnych atrybutów są niezależne od siebie.
Na tym polega ,,naiwność'' klasyfikatora Bayesa.

Naiwny klasyfikator Bayesa zostanie przetestowany przy pomocy funkcji \texttt{naiveBayes} z~pakietu \texttt{e1071} języka~R
\footnote{\url{http://www-users.cs.york.ac.uk/~jc/teaching/arin/R_practical/}}.

\subsection{SVM}
SVM jest klasyfikatorem binarnym, który ma na celu wyznaczenie hiperpłaszczyzny rozdzielającej 
z~maksymalnym marginesem przykłady należące do dwóch klas. W~ramach projektu realizowana będzie 
klasyfikacja mająca na celu przypisanie połączeń sieciowych do jednej z~pięciu klas. 
Problem klasyfikacji wieloklasowej może być zdekomponowany do szeregu problemów klasyfikacji binarnej. 
W~projekcie wykorzystane zostanie podejście \textit{one-against-one}, 
które polega na zbudowaniu klasyfikatora SVM dla każdej pary klas. 

Do przetestowania klasyfikatora SVM zostanie wykorzystana funkcja \texttt{svm} z~pakietu 
\texttt{e1071} języka~R \footnote{\url{http://cran.r-project.org/web/packages/e1071/}}.

\section{Przygotowanie danych}

Podczas realizacji projektu został wykorzystany zbiór danych z konkursu 
KDD'99 Classifier Learning Contest
\footnote{\url{www.sigkdd.org/kdd-cup-1999-computer-network-intrusion-detection}}.
Każdy przykład zawiera 41 atrybutów połączenia sieciowego. W poniższej tabeli przedstawiono
szczegóły wszystkich atrybutów.
42 kolumna danych treningowych zawiera typ ataku, który bezpośrednio mapuję się na
jedną z 5 ostatecznych klas. \\

\begin{tabular}{ | l | p{3cm} | p{3cm} | p{6cm} | } \hline
Nr & Nazwa & Opis & Zbiór wartości \\ \hline
1*      & duration & czas połączenia w sekundach & $\mathbb{Z}$ \\ \hline
2*      & protocol\_type & rodzaj protokołu & icmp, tcp, udp \\ \hline
3*      & service & usługa na serwerze & smtp, bgp, imap4, courier, name, exec, ftp, echo, http\_2784,
                       http\_443, discard, kshell, login, http, Z39\_50, vmnet, supdup,
                       gopher, printer, aol, tftp\_u, csnet\_ns, http\_8001, eco\_i, time,
                       ssh, efs, hostnames, X11, klogin, sql\_net, ldap, private,
                       auth, uucp, pm\_dump, link, ctf, IRC, ecr\_i, netbios\_ns, urp\_i,
                       pop\_2, pop\_3, rje, systat, ftp\_data,finger, tim\_i, remote\_job,
                       other, domain\_u, urh\_i, iso\_tsap, netstat, daytime, whois, shell,
                       mtp, sunrpc, uucp\_path, red\_i, harvest, nnsp, telnet, domain,
                       ntp\_u, netbios\_dgm, nntp, netbios\_ssn \\ \hline
4      & flag & status połączenia & REJ, SF, SH, RSTO, OTH, RSTR, RSTOS0, S0, S1, S2, S3 \\ \hline
5*      & src\_bytes  & bajty przesłane w połączeniu & $\mathbb{Z}$ \\ \hline
6*      & dst\_bytes  & bajty pobrane w połączeniu  & $\mathbb{Z}$ \\ \hline
7      & land & 1, jeśli źródło oraz cel połączenia ma tego samego hosta i port & 0, 1 \\ \hline
8      & wrong\_fragment  & liczba pakietów oznaczonych jako ,,wrong'' & $\mathbb{Z}$ \\ \hline
9      & urgent  & liczba pakietów oznaczonych jako ,,urgent''  & $\mathbb{Z}$ \\ \hline
\end{tabular}

\begin{tabular}{ | l | l | p{6,5cm} | p{1,5cm} | } \hline
Nr & Nazwa & Opis & Zbiór wartości \\ \hline
10      & hot & liczba wskaźników ,,hot'' dla akcji, np. uruchamianie programów & $\mathbb{Z}$ \\ \hline
11      & num\_failed\_logins  & liczba nieudanych prób logowania w połączeniu & $\mathbb{Z}$ \\ \hline
12*      & logged\_in  & 1, jeśli udało się zalogować &  0, 1 \\ \hline
13      & num\_compromised & liczba wystąpień błędu ,,not found'' podczas połączenia &  $\mathbb{Z}$ \\ \hline
14      & root\_shell  & 1, jeśli uzyskano konsolę z uprawnieniami root  &  0, 1 \\ \hline
15      & su\_attempted  & 1, jeśli użyto komendy \texttt{su} &  0, 1 \\ \hline
16      & num\_root  & liczba operacji wykonanych jako root  & $\mathbb{Z}$ \\ \hline
17      & num\_file\_creations  & liczba utworzonych plików  & $\mathbb{Z}$ \\ \hline
18      & num\_shells  & liczba zalogowań jako normalny użytkownik & $\mathbb{Z}$ \\ \hline
19      & num\_access\_files  & liczba operacji na plikach kontroli dostępu & $\mathbb{Z}$ \\ \hline
20      & num\_outbound\_cmds & liczba poleceń wychodzących w sesji FTP & $\mathbb{Z}$ \\ \hline
21      & is\_hot\_login  & 1, jeśli login należy to listy ,,hot'', tzn. jeśli login to root albo adm &  0, 1 \\ \hline
22      & is\_guest\_login  & 1, jeśli login należy to listy ,,guest'' &  0, 1 \\ \hline
23*     & count & liczba połączeń do tego samego hosta w przęciągu 2 sekund & $\mathbb{Z}$ \\ \hline
24*      & srv\_count   & liczba połączeń z tą samą usługą w przeciągu 2 sekund &  $\mathbb{Z}$ \\ \hline
25      & serror\_rate        & \% połączeń o tym samym hoście z błędami SYN  & $\mathbb{R}$ \\ \hline
26      & srv\_serror\_rate   & \% połączeń do tej samej usługi z błędami SYN & $\mathbb{R}$ \\ \hline
27      & rerror\_rate        & \% połączeń o tym samym hoście z błędami REJ   & $\mathbb{R}$ \\ \hline
28      & srv\_rerror\_rate   & \% połączeń do tej samej usługi z błędami REJ  & $\mathbb{R}$ \\ \hline
29*      & same\_srv\_rate     & \% połączeń do tej samej usługi  &  $\mathbb{R}$ \\ \hline
30      & diff\_srv\_rate     & \% połączeń do różnych usług &  $\mathbb{R}$ \\ \hline  
31*      & srv\_diff\_host\_rate   & \% połączeń do różnych hostów   & $\mathbb{R}$ \\ \hline
\end{tabular}

\begin{tabular}{ | l | l | p{5,4cm} | p{1,5cm} | } \hline
Nr & Nazwa & Opis & Zbiór wartości \\ \hline
32      & dst\_host\_count & liczba połączeń do tego samego adresu IP  &  $\mathbb{Z}$ \\ \hline
33      & dst\_host\_srv\_count  & liczba połączeń do tego samego portu docelowego   & $\mathbb{Z}$ \\ \hline
34      & dst\_host\_same\_srv\_rate  & \% połączeń do tej samej usługi w stosunku do liczby połączeń do tego samego adresu IP (atrybut 31) & $\mathbb{R}$ \\ \hline
35*      & dst\_host\_diff\_srv\_rate  & \% połączeń do tej różnych usług w stosunku do liczby połączeń do tego samego adresu IP (atrybut 31) & $\mathbb{R}$ \\ \hline
36*      & dst\_host\_same\_src\_port\_rate & \% połączeń do tego samego portu źródłowego w stosunku do liczby połączeń do tego samego portu docelowego (atrybut 32)  & $\mathbb{R}$ \\ \hline
37*      & dst\_host\_srv\_diff\_host\_rate & \% połączeń do różnych portów źródłowych w stosunku do liczby połączeń do tego samego portu docelowego (atrybut 32)  &  $\mathbb{R}$ \\ \hline

38*      & dst\_host\_serror\_rate       & \% połączeń o tym samym hoście z błędami SYN w stosunku do atrybutu 31        &  $\mathbb{R}$ \\ \hline  
39      & dst\_host\_srv\_serror\_rate  & \% połączeń o tym samym hoście z błędami SYN w stosunku do atrybutu 32 & $\mathbb{R}$ \\ \hline

40      & dst\_host\_rerror\_rate        & \% połączeń o tym samym hoście z błędami REJ w stosunku do atrybutu 31 & $\mathbb{R}$ \\ \hline
41      & dst\_host\_srv\_rerror\_rate   & \% połączeń o tym samym hoście z błędami REJ w stosunku do atrybutu 32 & $\mathbb{R}$ \\ \hline
\end{tabular}

\begin{tabular}{ | l | l | p{4,5cm} | p{5,3cm} | } \hline
Nr & Nazwa & Opis & Zbiór wartości \\ \hline
42      & attack\_type & rodzaj ataku sieciowego & back, buffer\_overflow, ftp\_write, guess\_passwd, imap, ipsweep, land, loadmodule, multihop, neptune, nmap, normal, perl, phf, pod, portsweep, rootkit, satan, smurf, spy, teardrop, warezclient, warezmaster \\ \hline
\end{tabular} \\\\

Znakiem gwiazdki (*) oznaczono 14 atrybutów wybranych do klasyfikacji.
Sposób w jaki je wybrano opisano w rozdziale \ref{sec:selekcja}.

\section{Statystyczny opis danych}

Zbiór testowy posiadał inny empiryczny rozkład prawdopodobieństwa typów ataków niż zbiór treningowy.
% TODO konkretne tabelki ile

\section{Transformacja danych}

Na początku dane zostały przejrzane w poszukiwaniu pojedynczych wartości odstających
powstałych np. na skutek błędu ludzkiego. Dokonano tego przy pomocy funkcji \texttt{summary}
języka R. Nie znaleziono wartości odstających. 
Następnie atrybuty wyliczeniowe zostały zamienione na liczbowe.

Wykorzystywane w~projekcie dane charakteryzują się nierównomiernym rozkładem klas. 
Dlatego przetestowano resampling przykładów treningowych, w~celu uzyskania równomiernego rozkładu klas. 
Wykorzystany został algorytm \texttt{SMOTE} udostępniany przez pakiet 
\texttt{DMwR} \footnote{\url{http://cran.r-project.org/web/packages/DMwR/}} języka~R.

\section{Selekcja atrybytów}
\label{sec:selekcja}

Wykorzystana została funkcja \texttt{importance} z pakietu lasów losowych \texttt{randomForest}.
\footnote{\url{http://cran.r-project.org/web/packages/randomForest/randomForest.pdf}}
Funkcja ta zwraca wartości dwóch miar ważności dla wszystkich atrybutów.
Pierwszą miarą jest średni spadek precyzji, drugą jest średni spadek nieczystości węzłów.

%TODO wykres i dane numeryczne

\section{Ocena jakości modeli}

Do porównywania jakości modeli wykorzystano błąd klasyfikacji na zbiorze testowym 
\texttt{corrected}\footnote{\url{http://www-cse.ucsd.edu/users/elkan/corrected.gz}},
przygotowanym przez autorów konkursu KDD'99.
Zbiór ten zawiera 311028 przykładów. 
Uczestnicy KDD'99 do oceny swoich rozwiązań używali tego samego zbioru testowego,
co umożliwia porównanie wyników otrzymanych w tej pracy z wynikami z konkursu.

\section{Wyniki eksperymentów}

\subsection{Algorytmy z domyślnymi wartościami parametrów}

\subsubsection{k-NN}

\subsubsection{Naiwny klasyfikator Bayesa}

\subsubsection{SVM}

\subsection{Strojenie wartości parametrów algorytmów}



\subsubsection{k-NN}
\begin{itemize}
	\item wartość parametru $k \in \{1, 3, 5, 7, 9\}$
\end{itemize}

\subsubsection{Naiwny klasyfikator Bayesa}
\begin{itemize}
	\item wygładzanie Laplace'a
\end{itemize}

\subsubsection{SVM}
\begin{itemize}
 \item rodzaj jądra oraz parametry charakterystyczne dla poszczególnych typów
 \item wagi poszczególnych klas
\end{itemize}

\subsection{SMOTE}

\section{Wnioski}

\end{document}