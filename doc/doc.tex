\documentclass[a4paper, 12pt]{article}
\usepackage [MeX] {polski}
\usepackage [utf8] {inputenc}
\usepackage {hyperref}
\usepackage{graphicx}
\usepackage[nottoc,numbib]{tocbibind}
\usepackage{color}
\usepackage{indentfirst}

\title {Metody odkrywania wiedzy - wykrywanie ataków sieciowych}
\author {Adam Stelmaszczyk\\ Anna Stępień}

\begin{document}

\maketitle

\tableofcontents

\newpage

%Klasyfikacja. Konkretyzacja tego zadania wymaga:
%ustalenia atrybutu dyskretnego reprezentującego pojęcie docelowe,
%określenia zakresu przygotowania danych (np. przetworzenia do odpowiedniej postaci tabelarycznej, modyfikacji typów/zbiorów wartości atrybutów, eliminacji/naprawy defektów danych, modyfikacji rozkładu kategorii, losowania podzbiorów danych),
%określenia zakresu i technik statystycznego opisu danych (np. charakterystyki %rozkładu wartości atrybytów, detekcji wartości odstających, detekcji zależności między atrybutami),
%wskazania możliwości zdefiniowania nowych atrybutów,
%ustalenia kryteriów lub algorytmu selekcji atrybutów,
%wyboru algorytmów klasyfikacji,
%wskazania parametrów algorytmów klasyfikacji wymagających strojenia,
%ustalenia procedur i kryteriów oceny jakości modeli (z uwzględnieniem rozkładu oraz, tam gdzie to uzasadnione, kosztów pomyłek).

\section{Opis projektu}

\paragraph{Dane}

\paragraph{Środowisko programowania}

\section{Analizowane algorytmy klasyfikacji}\label{algorithms}

\section{Sposób oceny jakości modelu}

\end{document}