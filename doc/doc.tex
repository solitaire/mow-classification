\documentclass[a4paper, 12pt]{article}
\usepackage[utf8]{inputenc}
\usepackage{polski}
\usepackage{hyperref}
\usepackage{graphicx}
\usepackage{indentfirst}

\title{Metody odkrywania wiedzy -- wykrywanie ataków sieciowych}
\author{Adam Stelmaszczyk\\ Anna Stępień}

\begin{document}

\maketitle

\tableofcontents

\newpage

\section{Interpretacja tematu projektu}
Projekt ma na celu analizę działania różnych algorytmów klasyfikacji zastosowanych
w procesie wykrywania ataków sieciowych. Połączenia sieciowe można podzielić na dwa typy:
\begin{itemize}
	\item prawidłowe,
	\item nieprawidłowe, naruszające bezpieczeństwo.
\end{itemize}
W ramach projektu zbadane zostaną algorytmy, które opisano w sekcji \ref{algorithms} 
tej pracy.

\section{Opis wykorzystanych algorytmów}\label{algorithms}

k-NN, SVM

\section{Plan badań}

\subsection{Pytania, na które będzie poszukiwana odpowiedź}

\begin{enumerate}
 \item Jakie parametry dla algorytmów dadzą najlepsze wyniki?
 \item Który algorytm przy najlepszych parametrach da lepsze wyniki?
 \item Który algorytm będzie działał szybciej?
 \item Które cechy miały największy wpływ na klasyfikację i dlaczego?
\end{enumerate}

\subsection{Analiza danych wejściowych}

Zbiorem danych, który zostanie wykorzystany podczas realizacji projektu jest zbiór danych dostępny 
pod adresem http://www.sigkdd.org/kdd-cup-1999-computer-network-intrusion-detection.

Wykorzystywany zbiór danych składa się ze zbiorów: treningowego i testowego. 
Zbiór testowy charakteryzuje się różnym rozkładem prawdopodobieństwa niż zbiór treningowy oraz 
zawiera więcej typów ataków sieciowych, które nie znajdują się w zbiorze treningowym.

Zdefiniowane w zbiorach danych typy ataków sieciowych mogą przynależeć do jednej z czterech kategorii: 
DOS, R2L, U2L, probing.

Na podstawie cech, połączenia zostaną zaklasyfikowane do jednej z pięciu grup:

\begin{enumerate}
 \item \texttt{normal} -- połączenia prawidłowe.
 \item \texttt{DOS} -- ataki typu Denial of Service.
 \item \texttt{R2L} -- nieautoryzowany dostęp ze zdalnej maszyny, np. zgadywanie hasła.
 \item \texttt{U2L} -- przejęcie praw administratora lokalnej maszyny, np. ataki przepełnienia bufora.
 \item \texttt{probing} -- odpytywanie, np. skanowanie portów.
\end{enumerate}

Na początku dane zostaną przejrzane w poszukiwaniu wartości odstających.
Przykłady odstające zostaną usunięte. Następnie zostanie wybranych kilka najlepiej rokujących
cech. Przetestowane zostanie też działanie algorytmów biorące wszystkie cechy pod uwagę.
Dane zostaną znormalizowane.

\subsection{Parametry algorytmów, których wpływ na wyniki będzie badany}

\begin{itemize}
 \item Wartość $k$ dla algorytmu k-NN.
 \item Coś (jądro?) dla SVM.
\end{itemize}

\subsection{Sposób oceny jakości modelu}
Jakość modelu zostanie poddana ocenie zgodnie z procedurą konkursową,
poprzez analizę jakości klasyfikacji i macierzy pomyłek.

% Ponadto zostanie wykorzystana graficzna reprezentacja zależności wyników typu \textit{true-positive} i \textit{false-positive} w postaci krzywej ROC.

\section{Otwarte kwestie wymagające późniejszego rozwiązania}


\end{document}